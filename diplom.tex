\documentclass[a4paper,14pt]{extreport} % Класс документа (report для дипломов)
\usepackage[utf8]{inputenc} % Кодировка
\usepackage[T2A]{fontenc} % Кириллическая кодировка
\usepackage[russian]{babel} % Язык документа
\usepackage{amsmath,amsfonts,amssymb} % Математические пакеты
\usepackage{graphicx} % Для вставки графики
\usepackage[colorlinks=true, urlcolor=blue, linkcolor=black]{hyperref} % Гиперссылки
\usepackage{geometry} % Настройка полей
\geometry{left=3cm, right=1.5cm, top=2cm, bottom=2cm} % Поля страницы

% Настройка стилей заголовков
\usepackage{titlesec}
\titleformat{\chapter}[display]
{\normalfont\bfseries\centering}{\chaptertitlename\ \thechapter}{20pt}{\Huge}
\titlespacing*{\chapter}{0pt}{-50pt}{40pt}

\begin{document}

% Титульный лист
\begin{titlepage}
    \begin{center}
        \vspace*{1cm}
        
        \large
        Министерство науки и высшего образования Российской Федерации \\
        Название университета \\
        Факультет такой-то \\
        Кафедра такая-то \\
        
        \vspace{2cm}
        
        \LARGE
        \textbf{Исследование влияния ионосферы на пропускную способность канала связи при двусторонней передачи информации} \\
        \large
        (Дипломная работа) \\
        
        \vspace{2cm}
        
        \begin{minipage}{0.4\textwidth}
            \begin{flushleft}
                \large
                \textbf{Выполнил:} \\
                Студент группы ХХ-ХХ \\
                Иванов И.И. \\
            \end{flushleft}
        \end{minipage}
        ~
        \begin{minipage}{0.4\textwidth}
            \begin{flushright}
                \large
                \textbf{Научный руководитель:} \\
                к.т.н., доцент \\
                Петров П.П. \\
            \end{flushright}
        \end{minipage}
        
        \vfill
        
        \large
        Город – 2023
    \end{center}
\end{titlepage}

% Оглавление
\tableofcontents

% Введение
\chapter*{Введение}
\addcontentsline{toc}{chapter}{Введение}
Рассмотрим распространение тестового сигнала [что нибудь из радиотехники] - гармонической плоской волны в ионосфере. Введем декартову систему координат . Ось  направим “вверх” по радиусу земли . 

% Основные главы
\chapter{Обзор литературы и постановка задачи}
\section{Анализ современного состояния проблемы}
Обзор существующих исследований по теме, анализ литературы.

\section {Цель}
Предполагается исследовать коэффициент отражения и прохождения при наклонном падении гармонической плоской волны как при распространении от земли к спутнику, так и в обратном направлении.
\section{Постановка задачи}
Среду распространения электромагнитной волны считаем плоскослоистой, линейной, изотропной (безынерционной), стационарной, безграничной. Свободные заряды отсутствуют. 
Будем пренебрегать сферичностью поверхности земли, то есть считаем ее локально плоскослоистой. 
Точку O выберем, где-нибудь в ионосфере. 
Пусть гармоническая плоская волна горизонтальной или вертикальной поляризации падает снизу вверх под углом , - волновой вектор. 
Систему координат выберем так, что волна падает в плоскости.

Распространение электромагнитной волны описывается уравнениями Максвелла [Тамм, Сивухин].

\chapter{Теоретическая часть}
\section{Основные понятия и определения}
Теоретические основы, необходимые для понимания работы.

$$
\nabla \times \mathbf{E} = - \frac{\partial \mathbf{B}}{\partial t}
$$

$$
\nabla \times \mathbf{H} = \mathbf{j} +\frac{\partial \mathbf{D}}{\partial t}
$$

$$
\operatorname{div} \mathbf{D} = \rho
$$

$$
\operatorname{div} \mathbf{B} = 0
$$

$$
U''_{zz} - q\frac{\varepsilon'(z)}{\varepsilon} + U'_{z}+k^2\varepsilon_{nach}(\tilde{\varepsilon}-sin^2\Theta_0)U=0
$$

$$
U''_{ss} - q\frac{\varepsilon'(s)}{\varepsilon} + U'_{s}+\eta_0^2(\tilde{\varepsilon}-sin^2\Theta_0)U=0
$$

$$
\eta_0:=k\sqrt{\varepsilon}l=\frac{w}{c}{\varepsilon}l
$$


\section{Методы исследования}
Описание используемых методов и подходов.

\chapter{Практическая часть}
\section{Разработка модели/алгоритма/системы}
Описание разработанного решения.

\section{Эксперименты и результаты}
Представление результатов, таблицы, графики.

\begin{table}[h]
\centering
\caption{Пример таблицы с результатами}
\begin{tabular}{|l|c|r|}
\hline
Параметр & Значение 1 & Значение 2 \\
\hline
Характеристика А & 10 & 15 \\
Характеристика Б & 20 & 25 \\
\hline
\end{tabular}
\end{table}

\begin{figure}[h]
    \centering
    \caption{Пример рисунка}
    \label{fig:example}
\end{figure}


% Заключение
\chapter*{Заключение}
\addcontentsline{toc}{chapter}{Заключение}
Краткое изложение основных результатов, выводы и перспективы дальнейших исследований.

% Список литературы
\begin{thebibliography}{99}
\addcontentsline{toc}{chapter}{Список литературы}
\bibitem{book1} Автор. Название книги. -- Город: Издательство, год. -- 255 с.
\bibitem{article1} Автор. Название статьи // Журнал. -- год. -- № X. -- С. XX-YY.
\end{thebibliography}

% Приложения
\appendix
\chapter{Первое приложение}
Текст приложения.

\chapter{Второе приложение}
Дополнительные материалы.

\end{document}